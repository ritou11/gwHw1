\documentclass[a4paper,12pt]{article}
\usepackage[noabs,notoc]{HaotianReport}

\title{第一次作业}
\author{刘昊天}
\authorinfo{电博181班, 2018310648}
\runninghead{高等电力网络分析}
\studytime{2018年11月}

\begin{document}
    \maketitle
    %\newpage
    \section{节点导纳矩阵的计算与分析}
    \subsection{题目1}
    \paragraph{题目描述} 请阅读参考程序1,理解并解读每一模块的功能与实现方法。试用自己的方法计算节点不定导纳矩阵Y0,并以地为参考节点生成导纳矩阵Y。
    \subsubsection{程序解读}
    \begin{enumerate}
      \item 导入MatPower的IEEE 9节点模型,并给出了模型信息。N为节点数量,b为支路数量。
      \begin{lstlisting}[language=matlab]
mpc = case9();
N = size(mpc.bus,1);
b = size(mpc.branch,1);
      \end{lstlisting}
      \item 生成不包含接地支路的导纳矩阵。\\
      首先生成节点支路关联矩阵A,使用稀疏矩阵进行存储。mpc结构体中的branch矩阵储存了网络支路的信息,每一行为一条支路,其中第一列与第二列分别存储了支路的首节点、尾节点。生成的方式,是在首节点(行号)与支路(列号)之间赋值1,在尾节点(行号)与支路(列号)之间赋值-1。\\
      其次生成支路导纳矩阵yb,方法是计算每一条支路的复导纳并将其放在矩阵的对角元上。branch矩阵的第三列、第四列分别为支路电阻、支路电感,导纳由\cref{eq:y}给出。\\
      \begin{equation}
        \label{eq:y}
        y_i = (R_i + j \cdot X_i)^{-1}, i\in[1, b]
      \end{equation}
      最后生成不包含接地支路的导纳矩阵Y。\\
      \begin{lstlisting}[language=matlab]
A = sparse(mpc.branch(:,1:2),[1:b,1:b]',[ones(b,1),-ones(b,1)]);
yb = spdiags(1./(mpc.branch(:,3)+1j*mpc.branch(:,4)),0,b,b);
Y = A*yb*A';
display(det(Y));
      \end{lstlisting}
      采用求取行列式的方式验证此时Y的奇异性,输出值为-2.0717e-05 + 1.3435e-05i。可见该值接近于0,Y矩阵此时是奇异的。这是由于没有接地支路,整个网络悬空,自然无法通过电流求取电压,符合物理意义。
      \item 补充接地支路,生成导纳矩阵Y。\\
      首先生成节点支路关联矩阵A,使用稀疏矩阵进行存储。由于为接地支路,因此节点与支路之间的关联元素均为1。构建A的目的,是为了从支路的对地导纳中等效出节点的接地导纳。
      其次使用A从支路上等效出节点的接地导纳,并生成节点接地导纳向量y0。每条线路使用$\pi$型等值电路,对地导纳等效在线路两侧,各1/2。同时每个节点有自身的对地导纳,储存在bus矩阵的第五列(电导)与第六列(电纳)中。
      最后,将接地导纳向量y0加到Y矩阵的对角元上,形成导纳矩阵Y。
      \begin{lstlisting}[language=matlab]
A = sparse(mpc.branch(:,1:2),[1:b,1:b]',ones(b,2));
y0 = A*1j*mpc.branch(:,5)/2 + mpc.bus(mpc.bus(:,1),5)+1j*mpc.bus(mpc.bus(:,1),6);
Y = Y + spdiags(y0,0,N,N);
      \end{lstlisting}
      \item 生成不定导纳矩阵Y0并测试奇异性
      \begin{lstlisting}[language=matlab]
Y0 = [Y,-y0;-y0.',sum(y0)];
display(det(Y0));
      \end{lstlisting}
      根据Y0的定义,将N+1节点定为大地节点,则可生成Y0阵。计算其行列式值,结果为6.7246e-06 + 8.5978e-06i。这表明Y0阵是奇异的,理由同样是没有参考节点。
    \end{enumerate}
    \subsubsection{不定导纳矩阵Y0生成}
    生成Y0矩阵时,不同于参考程序的思路,我们可以将大地节点作为N+1号节点,直接写出全网的节点支路关联矩阵,并计算全网的支路导纳。对于线路的对地导纳,我们可以用$\pi$形等值电路将导纳等效到节点上;对于节点的接地导纳,可以将其看做一条对地的支路。该思路生成Y0的程序如\cref{lst:q1p1}所示。
    \begin{lstlisting}[language=matlab,label=lst:q1p1,caption={不定节点导纳矩阵Y0生成程序}]
%% A0
nodeA = [mpc.branch(:,1);mpc.branch(:,2);(N+1)*ones(N,1);(1:N)'];
branchA = [1:b,1:b,b+(1:N),b+(1:N)]';
valueA = [ones(b,1);-ones(b,1);-ones(N,1);ones(N,1)];
A0 = sparse(nodeA, branchA, valueA);
%% yb
A = sparse(mpc.branch(:,1:2),[1:b,1:b]',ones(b,2));
y0 = A*1j*mpc.branch(:,5)/2 + mpc.bus(mpc.bus(:,1),5) + 1j*mpc.bus(mpc.bus(:,1),6);
yb = spdiags([1./(mpc.branch(:,3)+1j*mpc.branch(:,4));y0],0,b+N,b+N);
%% Y0 Y
Y0 = A0*yb*A0';
    \end{lstlisting}
    \subsubsection{导纳矩阵Y生成}
    由于大地节点是N+1号节点,因此直接截取Y0阵的前N行N列,即可得到导纳矩阵Y。
    \begin{lstlisting}[language=matlab]
Y = Y0(1:N, 1:N);
    \end{lstlisting}
    \newpage
    \appendix
    \section{程序清单}
      \begin{lstlisting}[language=matlab,label=lst:q1p2,caption={作业任务主程序}]
  A = sparse(mpc.branch(:,1:2),[1:b,1:b]',[ones(b,1),-ones(b,1)]);
  yb = spdiags(1./(mpc.branch(:,3)+1j*mpc.branch(:,4)),0,b,b);
  Y = A*yb*A';
  display(det(Y));
      \end{lstlisting}
    \label{applastpage}
    \newpage
    \bibliography{report}
    \bibliographystyle{unsrt}
\iffalse
\begin{itemize}[noitemsep,topsep=0pt]
%no white space
\end{itemize}
\begin{enumerate}[label=\Roman{*}.,noitemsep,topsep=0pt]
%use upper case roman
\end{enumerate}
\begin{multicols}{2}
%two columns
\end{multicols}
\fi
\end{document}
